% Options for packages loaded elsewhere
\PassOptionsToPackage{unicode}{hyperref}
\PassOptionsToPackage{hyphens}{url}
%
\documentclass[
]{article}
\usepackage{amsmath,amssymb}
\usepackage{iftex}
\ifPDFTeX
  \usepackage[T1]{fontenc}
  \usepackage[utf8]{inputenc}
  \usepackage{textcomp} % provide euro and other symbols
\else % if luatex or xetex
  \usepackage{unicode-math} % this also loads fontspec
  \defaultfontfeatures{Scale=MatchLowercase}
  \defaultfontfeatures[\rmfamily]{Ligatures=TeX,Scale=1}
\fi
\usepackage{lmodern}
\ifPDFTeX\else
  % xetex/luatex font selection
\fi
% Use upquote if available, for straight quotes in verbatim environments
\IfFileExists{upquote.sty}{\usepackage{upquote}}{}
\IfFileExists{microtype.sty}{% use microtype if available
  \usepackage[]{microtype}
  \UseMicrotypeSet[protrusion]{basicmath} % disable protrusion for tt fonts
}{}
\makeatletter
\@ifundefined{KOMAClassName}{% if non-KOMA class
  \IfFileExists{parskip.sty}{%
    \usepackage{parskip}
  }{% else
    \setlength{\parindent}{0pt}
    \setlength{\parskip}{6pt plus 2pt minus 1pt}}
}{% if KOMA class
  \KOMAoptions{parskip=half}}
\makeatother
\usepackage{xcolor}
\usepackage[margin=1in]{geometry}
\usepackage{color}
\usepackage{fancyvrb}
\newcommand{\VerbBar}{|}
\newcommand{\VERB}{\Verb[commandchars=\\\{\}]}
\DefineVerbatimEnvironment{Highlighting}{Verbatim}{commandchars=\\\{\}}
% Add ',fontsize=\small' for more characters per line
\usepackage{framed}
\definecolor{shadecolor}{RGB}{248,248,248}
\newenvironment{Shaded}{\begin{snugshade}}{\end{snugshade}}
\newcommand{\AlertTok}[1]{\textcolor[rgb]{0.94,0.16,0.16}{#1}}
\newcommand{\AnnotationTok}[1]{\textcolor[rgb]{0.56,0.35,0.01}{\textbf{\textit{#1}}}}
\newcommand{\AttributeTok}[1]{\textcolor[rgb]{0.13,0.29,0.53}{#1}}
\newcommand{\BaseNTok}[1]{\textcolor[rgb]{0.00,0.00,0.81}{#1}}
\newcommand{\BuiltInTok}[1]{#1}
\newcommand{\CharTok}[1]{\textcolor[rgb]{0.31,0.60,0.02}{#1}}
\newcommand{\CommentTok}[1]{\textcolor[rgb]{0.56,0.35,0.01}{\textit{#1}}}
\newcommand{\CommentVarTok}[1]{\textcolor[rgb]{0.56,0.35,0.01}{\textbf{\textit{#1}}}}
\newcommand{\ConstantTok}[1]{\textcolor[rgb]{0.56,0.35,0.01}{#1}}
\newcommand{\ControlFlowTok}[1]{\textcolor[rgb]{0.13,0.29,0.53}{\textbf{#1}}}
\newcommand{\DataTypeTok}[1]{\textcolor[rgb]{0.13,0.29,0.53}{#1}}
\newcommand{\DecValTok}[1]{\textcolor[rgb]{0.00,0.00,0.81}{#1}}
\newcommand{\DocumentationTok}[1]{\textcolor[rgb]{0.56,0.35,0.01}{\textbf{\textit{#1}}}}
\newcommand{\ErrorTok}[1]{\textcolor[rgb]{0.64,0.00,0.00}{\textbf{#1}}}
\newcommand{\ExtensionTok}[1]{#1}
\newcommand{\FloatTok}[1]{\textcolor[rgb]{0.00,0.00,0.81}{#1}}
\newcommand{\FunctionTok}[1]{\textcolor[rgb]{0.13,0.29,0.53}{\textbf{#1}}}
\newcommand{\ImportTok}[1]{#1}
\newcommand{\InformationTok}[1]{\textcolor[rgb]{0.56,0.35,0.01}{\textbf{\textit{#1}}}}
\newcommand{\KeywordTok}[1]{\textcolor[rgb]{0.13,0.29,0.53}{\textbf{#1}}}
\newcommand{\NormalTok}[1]{#1}
\newcommand{\OperatorTok}[1]{\textcolor[rgb]{0.81,0.36,0.00}{\textbf{#1}}}
\newcommand{\OtherTok}[1]{\textcolor[rgb]{0.56,0.35,0.01}{#1}}
\newcommand{\PreprocessorTok}[1]{\textcolor[rgb]{0.56,0.35,0.01}{\textit{#1}}}
\newcommand{\RegionMarkerTok}[1]{#1}
\newcommand{\SpecialCharTok}[1]{\textcolor[rgb]{0.81,0.36,0.00}{\textbf{#1}}}
\newcommand{\SpecialStringTok}[1]{\textcolor[rgb]{0.31,0.60,0.02}{#1}}
\newcommand{\StringTok}[1]{\textcolor[rgb]{0.31,0.60,0.02}{#1}}
\newcommand{\VariableTok}[1]{\textcolor[rgb]{0.00,0.00,0.00}{#1}}
\newcommand{\VerbatimStringTok}[1]{\textcolor[rgb]{0.31,0.60,0.02}{#1}}
\newcommand{\WarningTok}[1]{\textcolor[rgb]{0.56,0.35,0.01}{\textbf{\textit{#1}}}}
\usepackage{graphicx}
\makeatletter
\def\maxwidth{\ifdim\Gin@nat@width>\linewidth\linewidth\else\Gin@nat@width\fi}
\def\maxheight{\ifdim\Gin@nat@height>\textheight\textheight\else\Gin@nat@height\fi}
\makeatother
% Scale images if necessary, so that they will not overflow the page
% margins by default, and it is still possible to overwrite the defaults
% using explicit options in \includegraphics[width, height, ...]{}
\setkeys{Gin}{width=\maxwidth,height=\maxheight,keepaspectratio}
% Set default figure placement to htbp
\makeatletter
\def\fps@figure{htbp}
\makeatother
\setlength{\emergencystretch}{3em} % prevent overfull lines
\providecommand{\tightlist}{%
  \setlength{\itemsep}{0pt}\setlength{\parskip}{0pt}}
\setcounter{secnumdepth}{-\maxdimen} % remove section numbering
\ifLuaTeX
  \usepackage{selnolig}  % disable illegal ligatures
\fi
\IfFileExists{bookmark.sty}{\usepackage{bookmark}}{\usepackage{hyperref}}
\IfFileExists{xurl.sty}{\usepackage{xurl}}{} % add URL line breaks if available
\urlstyle{same}
\hypersetup{
  pdftitle={Preliminary Assessment of the Dataset},
  pdfauthor={John Ramírez, JR Engineering Company},
  hidelinks,
  pdfcreator={LaTeX via pandoc}}

\title{Preliminary Assessment of the Dataset}
\author{John Ramírez, JR Engineering Company}
\date{2024-04-17}

\begin{document}
\maketitle

\hypertarget{exploraciuxf3n-inicial-de-datos}{%
\subsection{Exploración Inicial de
Datos}\label{exploraciuxf3n-inicial-de-datos}}

\hypertarget{preliminares}{%
\subsubsection{Preliminares}\label{preliminares}}

Este chunk de código se ejecuta al principio pero no se incluye en el
documento final. Primero, establece la opción echo = TRUE para que todos
los códigos se muestren en el documento. Luego, carga la biblioteca
tidyverse, que contiene varias herramientas para análisis de datos.
Finalmente, lee el archivo de datos CSV llamado ``murders.csv'' y lo
guarda en el objeto ``data''.

\begin{Shaded}
\begin{Highlighting}[]
\NormalTok{knitr}\SpecialCharTok{::}\NormalTok{opts\_chunk}\SpecialCharTok{$}\FunctionTok{set}\NormalTok{(}\AttributeTok{echo =} \ConstantTok{TRUE}\NormalTok{)}
\FunctionTok{library}\NormalTok{(tidyverse)}
\end{Highlighting}
\end{Shaded}

\begin{verbatim}
## -- Attaching core tidyverse packages ------------------------ tidyverse 2.0.0 --
## v dplyr     1.1.4     v readr     2.1.5
## v forcats   1.0.0     v stringr   1.5.1
## v ggplot2   3.5.0     v tibble    3.2.1
## v lubridate 1.9.3     v tidyr     1.3.1
## v purrr     1.0.2     
## -- Conflicts ------------------------------------------ tidyverse_conflicts() --
## x dplyr::filter() masks stats::filter()
## x dplyr::lag()    masks stats::lag()
## i Use the conflicted package (<http://conflicted.r-lib.org/>) to force all conflicts to become errors
\end{verbatim}

\begin{Shaded}
\begin{Highlighting}[]
\NormalTok{data }\OtherTok{\textless{}{-}} \FunctionTok{read.csv}\NormalTok{(}\StringTok{"./data/murders.csv"}\NormalTok{)}
\end{Highlighting}
\end{Shaded}

\hypertarget{visualizaciuxf3n-inicial}{%
\subsubsection{Visualización Inicial}\label{visualizaciuxf3n-inicial}}

Se muestra las primeras filas del conjunto de datos con head(data) y la
estructura del conjunto de datos con str(data). Esto permite una
inspección inicial de los datos.

\begin{Shaded}
\begin{Highlighting}[]
\FunctionTok{head}\NormalTok{(data)}
\end{Highlighting}
\end{Shaded}

\begin{verbatim}
##        state abb region population total
## 1    Alabama  AL  South    4779736   135
## 2     Alaska  AK   West     710231    19
## 3    Arizona  AZ   West    6392017   232
## 4   Arkansas  AR  South    2915918    93
## 5 California  CA   West   37253956  1257
## 6   Colorado  CO   West    5029196    65
\end{verbatim}

\begin{Shaded}
\begin{Highlighting}[]
\FunctionTok{str}\NormalTok{(data)}
\end{Highlighting}
\end{Shaded}

\begin{verbatim}
## 'data.frame':    51 obs. of  5 variables:
##  $ state     : chr  "Alabama" "Alaska" "Arizona" "Arkansas" ...
##  $ abb       : chr  "AL" "AK" "AZ" "AR" ...
##  $ region    : chr  "South" "West" "West" "South" ...
##  $ population: int  4779736 710231 6392017 2915918 37253956 5029196 3574097 897934 601723 19687653 ...
##  $ total     : int  135 19 232 93 1257 65 97 38 99 669 ...
\end{verbatim}

\hypertarget{resumen-estaduxedstico}{%
\subsubsection{Resumen Estadístico}\label{resumen-estaduxedstico}}

Se proporciona un resumen estadístico de las variables numéricas en el
conjunto de datos utilizando la función summary().

\begin{Shaded}
\begin{Highlighting}[]
\FunctionTok{summary}\NormalTok{(data)}
\end{Highlighting}
\end{Shaded}

\begin{verbatim}
##     state               abb               region            population      
##  Length:51          Length:51          Length:51          Min.   :  563626  
##  Class :character   Class :character   Class :character   1st Qu.: 1696962  
##  Mode  :character   Mode  :character   Mode  :character   Median : 4339367  
##                                                           Mean   : 6075769  
##                                                           3rd Qu.: 6636084  
##                                                           Max.   :37253956  
##      total       
##  Min.   :   2.0  
##  1st Qu.:  24.5  
##  Median :  97.0  
##  Mean   : 184.4  
##  3rd Qu.: 268.0  
##  Max.   :1257.0
\end{verbatim}

\hypertarget{visualizaciuxf3n-de-variables-numuxe9ricas}{%
\subsubsection{Visualización de Variables
Numéricas}\label{visualizaciuxf3n-de-variables-numuxe9ricas}}

Se crea histogramas y diagramas de caja para las variables ``total''
(número total de homicidios) y ``population'' (población). Los
histogramas muestran la distribución de los datos, mientras que los
diagramas de caja muestran la dispersión y los valores atípicos.

\begin{Shaded}
\begin{Highlighting}[]
\DocumentationTok{\#\#\#\# Histogramas: Histograms provide a visual representation of the distribution of a numerical variable. They display the frequency of data points falling within certain intervals.}

\FunctionTok{hist}\NormalTok{(data}\SpecialCharTok{$}\NormalTok{total, }\AttributeTok{main =} \StringTok{"Histogram of Total Murders"}\NormalTok{, }\AttributeTok{xlab =} \StringTok{"Total Murders"}\NormalTok{, }\AttributeTok{ylab =} \StringTok{"Frequency"}\NormalTok{, }\AttributeTok{col =} \StringTok{"skyblue"}\NormalTok{)}
\end{Highlighting}
\end{Shaded}

\includegraphics{Untitled_files/figure-latex/vd-1.pdf}

\begin{Shaded}
\begin{Highlighting}[]
\FunctionTok{hist}\NormalTok{(data}\SpecialCharTok{$}\NormalTok{population }\SpecialCharTok{/} \DecValTok{10}\SpecialCharTok{\^{}}\DecValTok{6}\NormalTok{, }\AttributeTok{main =} \StringTok{"Histogram of Population (in millions)"}\NormalTok{, }\AttributeTok{xlab =} \StringTok{"Population (in millions)"}\NormalTok{, }\AttributeTok{ylab =} \StringTok{"Frequency"}\NormalTok{, }\AttributeTok{col =} \StringTok{"lightgreen"}\NormalTok{)}
\end{Highlighting}
\end{Shaded}

\includegraphics{Untitled_files/figure-latex/vd-2.pdf}

\begin{Shaded}
\begin{Highlighting}[]
\DocumentationTok{\#\#\#\# Boxplots: The boxplots provide a visual summary of the distribution of the data. They display the median, quartiles, and potential outliers.}

\FunctionTok{boxplot}\NormalTok{(data}\SpecialCharTok{$}\NormalTok{total, }\AttributeTok{main =} \StringTok{"Boxplot of Total Murders"}\NormalTok{, }\AttributeTok{ylab =} \StringTok{"Total Murders"}\NormalTok{, }\AttributeTok{col =} \StringTok{"salmon"}\NormalTok{)}
\end{Highlighting}
\end{Shaded}

\includegraphics{Untitled_files/figure-latex/vd-3.pdf}

\begin{Shaded}
\begin{Highlighting}[]
\FunctionTok{boxplot}\NormalTok{(data}\SpecialCharTok{$}\NormalTok{population }\SpecialCharTok{/} \DecValTok{10}\SpecialCharTok{\^{}}\DecValTok{6}\NormalTok{, }\AttributeTok{main =} \StringTok{"Boxplot of Population (in millions)"}\NormalTok{, }\AttributeTok{ylab =} \StringTok{"Population (in millions)"}\NormalTok{, }\AttributeTok{col =} \StringTok{"lightblue"}\NormalTok{)}
\end{Highlighting}
\end{Shaded}

\includegraphics{Untitled_files/figure-latex/vd-4.pdf}

\hypertarget{visualizaciuxf3n-de-relaciones}{%
\subsubsection{Visualización de
Relaciones}\label{visualizaciuxf3n-de-relaciones}}

Se crea un gráfico de dispersión para explorar la relación entre
``total'' (número total de homicidios) y ``population'' (población). Se
divide la población por 10\^{}5 para escalar los valores a un rango más
manejable.

\begin{Shaded}
\begin{Highlighting}[]
\FunctionTok{plot}\NormalTok{(data}\SpecialCharTok{$}\NormalTok{population}\SpecialCharTok{/}\DecValTok{10}\SpecialCharTok{\^{}}\DecValTok{5}\NormalTok{, data}\SpecialCharTok{$}\NormalTok{total, }\AttributeTok{xlab =} \StringTok{"Population (in hundreds of thousands)"}\NormalTok{, }
     \AttributeTok{ylab =} \StringTok{"Total Murders"}\NormalTok{,}
     \AttributeTok{main =} \StringTok{"Scatter plot of Total Murders vs Population"}\NormalTok{,}
     \AttributeTok{col =} \StringTok{"darkblue"}\NormalTok{,}
     \AttributeTok{pch =} \DecValTok{19}\NormalTok{)}
\end{Highlighting}
\end{Shaded}

\includegraphics{Untitled_files/figure-latex/vr-1.pdf}

\hypertarget{tabla-de-contingencia}{%
\subsubsection{Tabla de Contingencia}\label{tabla-de-contingencia}}

Se crea una tabla de contingencia que muestra la frecuencia de
combinaciones entre las variables ``abb'' (abreviaturas de estados) y
``region'' (región geográfica).

\begin{Shaded}
\begin{Highlighting}[]
\FunctionTok{table}\NormalTok{ (data}\SpecialCharTok{$}\NormalTok{abb, data}\SpecialCharTok{$}\NormalTok{region)}
\end{Highlighting}
\end{Shaded}

\begin{verbatim}
##     
##      North Central Northeast South West
##   AK             0         0     0    1
##   AL             0         0     1    0
##   AR             0         0     1    0
##   AZ             0         0     0    1
##   CA             0         0     0    1
##   CO             0         0     0    1
##   CT             0         1     0    0
##   DC             0         0     1    0
##   DE             0         0     1    0
##   FL             0         0     1    0
##   GA             0         0     1    0
##   HI             0         0     0    1
##   IA             1         0     0    0
##   ID             0         0     0    1
##   IL             1         0     0    0
##   IN             1         0     0    0
##   KS             1         0     0    0
##   KY             0         0     1    0
##   LA             0         0     1    0
##   MA             0         1     0    0
##   MD             0         0     1    0
##   ME             0         1     0    0
##   MI             1         0     0    0
##   MN             1         0     0    0
##   MO             1         0     0    0
##   MS             0         0     1    0
##   MT             0         0     0    1
##   NC             0         0     1    0
##   ND             1         0     0    0
##   NE             1         0     0    0
##   NH             0         1     0    0
##   NJ             0         1     0    0
##   NM             0         0     0    1
##   NV             0         0     0    1
##   NY             0         1     0    0
##   OH             1         0     0    0
##   OK             0         0     1    0
##   OR             0         0     0    1
##   PA             0         1     0    0
##   RI             0         1     0    0
##   SC             0         0     1    0
##   SD             1         0     0    0
##   TN             0         0     1    0
##   TX             0         0     1    0
##   UT             0         0     0    1
##   VA             0         0     1    0
##   VT             0         1     0    0
##   WA             0         0     0    1
##   WI             1         0     0    0
##   WV             0         0     1    0
##   WY             0         0     0    1
\end{verbatim}

\hypertarget{visualizaciuxf3n-con-ggplot2}{%
\subsubsection{Visualización con
ggplot2}\label{visualizaciuxf3n-con-ggplot2}}

Se utiliza la biblioteca ggplot2 para crear una visualización de la
relación entre ``abb'' (abreviaturas de estados) y ``total'' (número
total de homicidios). Primero, se realizan algunas transformaciones en
los datos, como convertir la variable ``region'' en un factor, calcular
la tasa de homicidios por cada 100,000 habitantes y reordenar las
abreviaturas de estado según la tasa de homicidios.

Luego, se crea un gráfico de dispersión con ggplot(), donde los puntos
representan cada estado y la tasa de homicidios se muestra en el eje y.
Se aplican algunas personalizaciones de estilo, como rotar las etiquetas
del eje x y voltear los ejes y colocar leyendas para cada eje y un
titulo para la gráfica.

\begin{Shaded}
\begin{Highlighting}[]
\NormalTok{data }\OtherTok{\textless{}{-}}\NormalTok{ data }\SpecialCharTok{\%\textgreater{}\%} \FunctionTok{mutate}\NormalTok{(}\AttributeTok{region =} \FunctionTok{factor}\NormalTok{ (region), }\AttributeTok{rate =}\NormalTok{ total }\SpecialCharTok{/}\NormalTok{ population }\SpecialCharTok{*} \DecValTok{10}\SpecialCharTok{\^{}}\DecValTok{5}\NormalTok{) }\SpecialCharTok{\%\textgreater{}\%} \FunctionTok{mutate}\NormalTok{(}\AttributeTok{abb =} \FunctionTok{reorder}\NormalTok{(abb, rate))}

\FunctionTok{ggplot}\NormalTok{(data, }\FunctionTok{aes}\NormalTok{ (}\AttributeTok{x=}\NormalTok{abb, }\AttributeTok{y=}\NormalTok{rate )) }\SpecialCharTok{+} \FunctionTok{geom\_point}\NormalTok{(}\AttributeTok{color =} \StringTok{"darkred"}\NormalTok{, }\AttributeTok{size =} \DecValTok{3}\NormalTok{, }\AttributeTok{alpha =} \FloatTok{0.7}\NormalTok{) }\SpecialCharTok{+}   
  \FunctionTok{theme}\NormalTok{(}\AttributeTok{axis.text.x =} \FunctionTok{element\_text}\NormalTok{(}\AttributeTok{angle =} \DecValTok{90}\NormalTok{, }\AttributeTok{vjust =} \FloatTok{0.5}\NormalTok{, }\AttributeTok{hjust =} \DecValTok{1}\NormalTok{)) }\SpecialCharTok{+} 
  \FunctionTok{coord\_flip}\NormalTok{() }\SpecialCharTok{+}
  \FunctionTok{labs}\NormalTok{(}\AttributeTok{x =} \StringTok{"State Abbreviations"}\NormalTok{, }
       \AttributeTok{y =} \StringTok{"Murder Rate (per 100,000 people)"}\NormalTok{,}
       \AttributeTok{title =} \StringTok{"Murder Rate Variation Across States"}\NormalTok{)}
\end{Highlighting}
\end{Shaded}

\includegraphics{Untitled_files/figure-latex/vis-1.pdf}

\hypertarget{referencias}{%
\subsubsection{Referencias}\label{referencias}}

Dataset:
``\url{https://raw.githubusercontent.com/rafalab/dslabs/master/inst/extdata/murders.csv}''

\end{document}
